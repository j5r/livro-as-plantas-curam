\documentclass[portuguese,12pt,a4paper,oneside,openright]{book}
% https://latexbr.blogspot.com/2013/01/indice-remissivo-no-latex.html
\usepackage{asplantascuram}
\title{As plantas curam}
\author{Alfons Balbach\\[1cm]\small\it Editado por Junior R. Ribeiro}
\def\l{$\ell$\space}
\begin{document}



\PLANTA{Abútua}{url}
\NOMECIENTIFICO{
    \cnome{cissampelos pareira}
    \cnome{cissampelos vitis}
}
\FAMILIA{menispermáceas.}
\OUTROSNOMES{parreira-brava, parreira-do-mato, uva-do-rio-apa, bútua, abuta.}
\USOMEDICINAL{
    \umed{cálculos renais} é diurética;
    \umed{cólicas uterinas} indicada contra cólicas que podem aparecer durante o sobreparto, contra a menstruação difícil e a supressão dos lóquios;
    \umed[má digestão]{dispepsia} é eficaz contra dispepsia acompanhada de prisão de ventre, dor de cabeça, tontura, etc.
    \umed{fígado} provoca a desopilação (desobstrução) nas afecções hepáticas;
    \umed{hidropisia},
    \umed{reumatismo}.
}
\PARTEUSADA{raiz e casca do tronco, por decocção.}
\DOSE{10 a 15g por \l de água; 4 a 5 xícaras por dia.}



%%%%%%%%%%%%%%%%%%%%
\PLANTA{abútua-miúda}{url}
\NOMECIENTIFICO{
    \cnome{cocculus filipendula}
}
\FAMILIA{menispermáceas}
\OUTROSNOMES{bútua-miúda}
\USOMEDICINAL{
    \umed{febre},
    \umed[ausência da menstruação]{amenorreia},
    \umed{\Pesquisa[clorose humana]{clorose}},
    \umed{cólical menstrual},
    \umed[útero]{\Pesquisa{metrite}}
}
\PARTEUSADA{casca e raiz por decocção.}
\DOSE{10 a 15 g por \l de água; 4 a 5 xícaras por dia.}
% \AVISOS{}



%%%%%%%%%%
\PLANTA{acariçoba}{url}
\NOMECIENTIFICO{
    \cnome{hydrocotyle umbellata}
    \cnome{hydrocotyle bonariensis}
}
\FAMILIA{umbelíferas}
\OUTROSNOMES{erva-do-capitão, barbarosa, acaciroba, acaricaba}
\USOMEDICINAL{
    \umed{\Pesquisa{aperiente}},
    \umed{\Pesquisa[significado remédio desobstruente]{desobstruente}},
    \umed{diurético}
    \umed{\Pesquisa[significado remédio tônico]{tônico}}
    \grupo{
        \umed{baço} \umed{fígado} \umed{intestino} \umed{diarreia} \umed{hidropisia} \umed{reumatismo} \umed{sífilis}: tomar o decocto da raiz.
    }
}
\PARTEUSADA{toda a planta.}
\DOSE{normal.}
\AVISOS{Não utilizar folhas para ingestão, apenas para uso externo.}





%%%%%%%%%
%%%%%%%%%
%%%%%%%%%%%%%%%%%%%%%%%%%%%%%%%%%%%%% FINAL
% a ultima planta da lista
\ImprimirPlanta 
\backmatter
\phantomsection
\cleardoublepage
\addcontentsline{toc}{chapter}{Índice de doenças}
\printindex[doencasix]
\addcontentsline{toc}{chapter}{Índice de plantas}
\printindex[plantasix]


\end{document}
